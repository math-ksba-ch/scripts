% [[file:../index.org::*Preamble][Preamble:1]]
\usepackage{geometry}
\geometry{a4paper, top=25mm, left=34mm, right=34mm, bottom=25mm}

\usepackage{emptypage}

\usepackage[titletoc,title]{appendix}
\usepackage{nameref}
\usepackage{amsfonts}
\usepackage{amsmath,amscd}
\usepackage{amsthm}
\usepackage{amssymb}
\usepackage{mathtools}
\usepackage{enumitem}
\usepackage[english]{babel}
\usepackage[utf8]{inputenc}
\usepackage{graphicx}
\usepackage{tikz}
\usepackage[framemethod=TikZ]{mdframed}
\tikzstyle{help lines}+=[dotted]
\usepackage{pgfplots} %To draw graphs of function directly in the latex file
%\usepackage{forest} %To create trees
%\usepackage{hyperref} %produce hypertext links
%\usepackage{pdfpages} % Include parts of pdf-files
\usepackage{fancyhdr}
\usepackage{tcolorbox}
\usepackage{comment} %let latex skip everything between \begin{comment} and \end{comment}


\bibliographystyle{plain}


%New command
\newcommand{\vsp}{\vspace{5mm}}
\newcommand{\hsp}{\hspace{5mm}}
\newcommand{\R}{\mathbb{R}}
\newcommand{\N}{\mathbb{N}}
\newcommand{\Z}{\mathbb{Z}}
\newcommand{\Q}{\mathbb{Q}}
\newcommand{\D}{\mathbb{D}}
\newcommand{\W}{\mathbb{W}}


% Theorem Style
\theoremstyle{plain}
\newtheorem{theorem}{Theorem}[section]
\newtheorem{lemma}[theorem]{Lemma}
\newtheorem{proposition}[theorem]{Proposition}
\newtheorem{corollary}[theorem]{Corollary}
\newtheorem*{theorem*}{Theorem}

\theoremstyle{definition}
\newtheorem{definition}[theorem]{Definition}
%\newtheorem{exer}{Exercise}   %in the added section 
%\newtheorem{exer*}[exer]{Exercise*}
\newtheorem*{example}{Example}

\theoremstyle{remark}
\newtheorem{remark}[theorem]{Remark}
\newtheorem*{remark*}{Remark}
\newtheorem{task}{Task}


%Equation numbering
%\numberwithin{equation}{section} 
%\renewcommand{\theequation}{\arabic{section}.\arabic{equation}}


%Redefine the first level
\renewcommand{\theenumi}{\roman{enumi}}

\renewcommand{\baselinestretch}{1.00}\normalsize

\renewcommand{\labelenumi}{\alph*)} % In enumerate wird a),b),c) etc. gezählt.


\parindent0mm
% Preamble:1 ends here

% [[file:../index.org::*Added][Added:1]]
\graphicspath{{images}}
\newif\ifEN\ENtrue
\def\tr|#1|#2|{\ifEN #1\else #2\fi}
\def\trx;#1;#2;{\ifEN #1\else #2\fi}


\newif\ifANSWERS\ANSWERSfalse% Mit Lösungen?
\newif\ifAnswersAtEnd\AnswersAtEndfalse	% Mit Lösungen am Ende?
%%%%%%%%%% Lösungen zu verschiedenen Aufgabentypen und Fragen im Theorieteil %%%%%%%%%%%%%%%%%%%%%%%%%%%%%%%%%%%%%%%%%%%
\usepackage[lastexercise,answerdelayed]{exercise}
% lastexercise: Lösungen beziehen sich jeweils direkt auf die Aufgabe davor 
% answerdelayed: Alle angesammelten Lösungen werden mit \shipoutAnswer ausgegeben
% 

\newcounter{exer}
%\newcounter{exc}[section]
%\renewcommand{\theexc}{\arabic{subsection}-\arabic{exc}}%Doesn't work
%\newcounter{refl}
%\newcounter{discover}
%\newcounter{research}
%\tr|\newenvironment{discover}{\begin{Exercise}[title={Entdecken},counter={discover}]}{\end{Exercise}}
%   |\newenvironment{discover}{\begin{Exercise}[title={Discover},counter={discover}]}{\end{Exercise}}|
%\tr|\newenvironment{research}{\begin{Exercise}[title={Erforschen},counter={research}]}{\end{Exercise}}
%   |\newenvironment{research}{\begin{Exercise}[title={Research},counter={research}]}{\end{Exercise}}|
\tr|\newenvironment{exer}{\begin{Exercise}[title={Exercise}, counter={exer}]} {\end{Exercise}}
   |\newenvironment{exer}{\begin{Exercise}[title={Übung}, counter={exer}]} {\end{Exercise}}|
%\newenvironment{refl}{\begin{Exercise}[title={Reflection}, counter={refl}]} {\end{Exercise}}
%\newenvironment{intro}{\begin{Exercise}[title={Introductory problem}, counter={refl}]} {\end{Exercise}}
%Too complicated:
%\renewcommand{\ExerciseHeader}{\textbf{\ExerciseTitle\ \arabic{section}.\arabic{subsection}-\ExerciseHeaderNB }}
%\renewcommand{\AnswerHeader}{\textbf{Solution of \ExerciseTitle\  \arabic{section}.\arabic{subsection}-\ExerciseHeaderNB\  }}
%Simpler
\renewcommand{\ExerciseHeader}{\textbf{\ExerciseTitle\ \ExerciseHeaderNB }}
\renewcommand{\AnswerHeader}{\textbf{\ExerciseTitle\  \ExerciseHeaderNB\  }}



%Lösungen zu...:
%\begin{exc}
%\end{exc}
%\begin{Answer}
%\end{Answer}

%Eine Lösung zu einer Aufgabe im Text, kann folgendermassen angegeben werden
%\begin{Exercise*}[label={intext1}]
%\end{Exercise*}
%\begin{Answer}Seite \pageref{intext1}
%\end{Answer}

%Die Antworten kommen entweder alle am Ende oder dort wo die jeweiligen \myshipoutAnswer Befehle stehen. 
%\newif\ifAnswersAtEnd\AnswersAtEndtrue
\ifANSWERS 
  \ifEN
    \ifAnswersAtEnd\AtEndDocument{\newpage\section{Answers}\shipoutAnswer}
    \else\fi
  \else
    \ifAnswersAtEnd\AtEndDocument{\newpage\section{Lösungen}\shipoutAnswer}
    \else\fi
  \fi 
\fi

\setlength{\fboxsep}{0pt}
\definecolor{blank}{gray}{1.00}
\def\answerline#1{%
  \ifhmode\\[2ex]\fcolorbox{blank}{blank}{\hbox to \textwidth{\vbox to #1\baselineskip{}}}%
  \else\par\medskip\fcolorbox{blank}{blank}{\hbox to \textwidth{\vbox to #1\baselineskip{}}}\par\bigskip%
  \fi
}	
\def\flushanswer#1{\hfill\fcolorbox{blank}{blank}{\hbox to #1\linewidth{\vbox to \baselineskip{}}}}	
\def\answer#1{\raisebox{-0.3em}{\fcolorbox{blank}{blank}{\hbox to #1\linewidth {\vbox to 0.8\baselineskip{}}}}	}
\def\answerabs#1{\raisebox{-0.3em}{\fcolorbox{blank}{blank}{\hbox to #1 {\vbox to 0.8\baselineskip{}}}}	}



\usepackage{hyperref}
% Added:1 ends here
